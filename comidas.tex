\documentclass[%
a4paper,
twoside,
14pt
]{book}

% encoding, font, language
\usepackage[T1]{fontenc}
\usepackage[utf8]{inputenc}
\usepackage{lmodern}
\usepackage[spanish, english]{babel}

\usepackage{nicefrac}

\usepackage[
    handwritten,
    nowarnings,
    %myconfig
]
{xcookybooky}

\usepackage{blindtext}    % only needed for generating test text

\DeclareRobustCommand{\textcelcius}{\ensuremath{^{\circ}\mathrm{C}}}


\setcounter{secnumdepth}{1}
\renewcommand*{\recipesection}[2][]
{%
    \subsection[#1]{#2}
}
\renewcommand{\subsectionmark}[1]
{% no implementation to display the section name instead
}


\usepackage{hyperref}    % must be the last package
\hypersetup{%
    pdfauthor            = {Julio Eneriz},
    pdftitle             = {Prueba de recetas},
    pdfsubject           = {Recipes},
    pdfkeywords          = {example, recipes, cookbook, xcookybooky},
    pdfstartview         = {FitV},
    pdfview              = {FitH},
    pdfpagemode          = {UseNone}, % Options; UseNone, UseOutlines
    bookmarksopen        = {true},
    pdfpagetransition    = {Glitter},
    colorlinks           = {true},
    linkcolor            = {black},
    urlcolor             = {blue},
    citecolor            = {black},
    filecolor            = {black},
}

\hbadness=10000	% Ignore underfull boxes

\begin{document}

\title{Examples for using \textbf{xcookybooky}}
\author{Julio Eneriz\\ \href{mailto:recetas@jeneriz.com}{recetas@jeneriz.com}}
\maketitle

\noindent I saw this package (https://github.com/SvenHarder/xcookybooky/) browsing the web and got curious. I'm just trying to write a couple of recipes in Spanish to see if I like it and to pass them to some friends that are interested. I'm not even bothering to remove all the text since I can commit only 1h tonight to this project. Let's see if I can come back.

\noindent I was spending too much time thinking about how to organize it. Should I put all the recipes that have replacements for allergies in a specific section? I was also struggling with pizza and bread, since I can't talk about one without talking about the other... But it's better to have something to iterate even if I got everything wrong, at least I'll have something to fix.

http://davetrott.co.uk/2009/11/shake-things-up/

    \noindent The examples in this document require at least version~1.4 of the \texttt{xcookybooky}\footnote{\url{http://www.ctan.org/pkg/xcookybooky}} package. For more examples and test recipes especially for using hook functions take a look at the source files located at \url{https://code.google.com/p/xcookybooky/}. If you are interested in modifying the layout of \texttt{xcookybooky} you will find examples in the documentation as well as in the configuration file \textbf{xcookybooky.cfg}.

\tableofcontents

\vspace{5em}
\chapter{Recetas o comidas}
\section{Importa?}
Una tabla de quesos necesita una receta? 

\chapter{Sopas}
\section{¿Qué es una sopa?}
Originariamente, una sopa era pan remojado. Al menos, eso dijo Ibán Yarza en el comidista (\url{http://elcomidista.elpais.com/elcomidista/2016/11/02/receta/1478119013_283306.htm}) luego evolucionó. Los americanos usan "soup" para cualquier plato de cuchara. A mí me cuesta pensar que unas lentejas son una sopa... al menos, cuando estoy en España.

\section{Recetas}

\begin{recipe}
[% 
    preparationtime = {\unit[10]{minutos}},
    portion = {\portion{3-4}},
    source = {Nunca lo he visto por escrito}
]
{Sopa de tomate}
    { pictures
        small=pic/tomatoSoup,
    }
 
    \introduction{%
	Común en Estados Unidos o en Bélgica, pero no la he visto nunca en España.
    }
    
    \ingredients{%
	\unit[1000]{g} & Dos botes de tomate entero pelado, sin escurrir.  \\
	\unit[20]{g} & Aceite de oliva virgen extra \\
	1 & cebolla mediana (130g) \\
	1 & Pizca de sal \\
    }
    
    \preparation{%
	\step Pelar y cuartear la cebolla y picarla 2 segundos al 5 en la thermomix.
	\step Añadir el aceite y freir durante 4 minutos en temperatura Varoma, velocidad cuchara.
	\step Añadir el tomate, que bajará considerablemente la temperatura. Triturar durante 30 segundos al máximo para que no queden ni semillas. Añadir la sal.
	\step Cocinar 10 minutos en temperatura varoma y velocidad 3.
        \step Servir en platos adornar con un poco de aceite por encima. Se puede adornar con sal en escamas, comino, pimenton...
    }
    
    \suggestion[Croutons de pesto de aguacate]
    {%
	Si el pan le va bien a cualquier sopa, pan untado con pesto, todavía mejor.
    }
    
    \suggestion{%
	Especias habituales son la albahaca fresca, añadida al final, o la pimienta negra, al principio.
    }
    
    \suggestion{%
	Cambiar parte del líquido de los tomates por un caldo añadirá más complejidad a la sopa.
    }
    
    \hint{%
        Si en lugar de estar 10 minutos, está 2 horas, deja de llamarse sopa de tomate y se convierte en tomate frito.
    }
    
\end{recipe}



\chapter{Legumbres}
\section{¿Quién da más por menos?}
Baratas, ricas, nutritivas y agradecidas en la cocina.

\section{Consideraciones previas}
La mayoría hay que hidratarlas o compralas ya cocidas

\section{Recetas}

\begin{recipe}
[% 
    preparationtime = {\unit[10]{minutos}},
    portion = {\portion{3-4}},
    source = {Nunca lo he visto por escrito}
]
{Hummus sin tahina}
    
    \introduction{%
El hummus no es hummus sin tahina, son garbanzos aliñados. Al menos es lo que creía hasta que un día en MasterChef alguien se olvido de coger tahina y tuvo que improvisar... y uso lentejas en su lugar. Dudo mucho que fuese un accidente si no un artificio para presentar una idea sorprendente. El caso es que me quede con el detalle y, cuando años más tarde tuve que dejar de hacer el hummus normal por una de las alergias de mi hija, me vino bien recordarlo.
    }
    
    \ingredients{%
	\unit[570]{g} & Un bote de garbanzos, con el liquido (¡no se tira! ver seccion de aquafaba).  \\
	\unit[125]{g} & Lentejas cocidas  \\
	\unit[30]{g} & Aceite de oliva virgen extra \\
	\unit[30]{g} & Zumo de limón \\
	1 & Diente de ajo \\
	2 & Pizcas de cominos \\
	1 & Pizca de sal \\
    }
    
    \preparation{%
        \step Triturar todo en la batidora o Thermomix. Yo la pongo 10 segundos al 4 y luego 1 minuto al 7.
        \step Probar, ver la textura y ajustar. Si esta muy espeso, se puede poner más agua o limón. Rectificar la sal y cominos. Poner más lentejas si sabe mucho a garbanzo.
        \step Servir en una fuente y adornar con un poco de aceite por encima. Se puede adornar con sal en escamas, comino, pimenton...
    }
    
    \suggestion[Ajos y limón]
    {%
	Algunas recetas no ponen ajo y otras ponen 6 dientes de ajo... Demasiado rango para que dependa del gusto de cada uno. Kenji explico un dia que el sabor de ajo depende mucho de la acidez del entorno. Con un ajo esta bien, con 6 solo se puede tomar si segun se pelan, se dejan unos minutos en el limón. Daran mucho sabor pero no repetiran y no estara demasiado fuerte.
    }
    
    \suggestion{%
	Las lentejas suelo cocerlas, los garbanzos van de bote. Las lentejas también pueden ir de bote, pero dado que se hacen en un momento, no me supone tanto problema como tener que hidratar los garbanzos. Los garbanzos de bote o lata estan buenos y no hay mucha diferencia, aunque si es cierto que esta algo mejor con garbanzos cocidos de la forma tradicional.
    }
    
    \suggestion{%
	Hay gente que pela los garbanzos para que quede más fino. Una vez que coci yo los garbanzos intente apartar algunas pieles, lo cual es más facil si se sobrecuecen. Estaba algo más suave, pero no se si por cocer de más o por pelar y no creo que me moleste en averiguarlo nunca.
    }
    
    \hint{%
        Se puede comer con pan o con verduras, a mi me gusta mucho usar calabacín a la plancha. Hay quien lo usa para bocadillos.
    }
    
\end{recipe}

\begin{recipe}
[% 
    preparationtime = {\unit[10]{minutos}},
    portion = {\portion{3-4}},
    source = {Nunca lo he visto por escrito}
]
{Hummus bi tahina}
    
    \introduction{%
	Ojo, la tahina es pasta de sésamo, cuidado con las alergias
    }
    
    \ingredients{%
	\unit[570]{g} & Un bote de garbanzos, con el liquido (¡no se tira! ver seccion de aquafaba).  \\
	\unit[125]{g} & Lentejas cocidas  \\
	\unit[30]{g} & Aceite de oliva virgen extra \\
	\unit[30]{g} & Zumo de limón \\
	1 & Diente de ajo \\
	2 & Pizcas de cominos \\
    }
    
    \preparation{%
        \step Ver la receta sin tahina y hacer lo mismo
    }
        
\end{recipe}

\chapter{Pizzas y panes}
\section{Introducción}
\section{¿Cómo hornear pan en casa?}
\subsection{Los 12 pasos del pan}
\subsection{El panadero práctico}
\subsection{Autólisis}
\subsection{Pliegues}
\subsection{El horno dentro del horno}
\section{¿Cómo hornear pizza en casa?}
\section{Recetas}
\begin{recipe}
[% 
    preparationtime = {\unit[1]{h}},
    bakingtime={\unit[1]{h}},
    bakingtemperature={\protect\bakingtemperature{
        fanoven=\unit[230]{\textcelcius},
        topbottomheat=\unit[195]{C},
        topheat=\unit[195]{C},
        gasstove=Level 2}},
    portion = {\portion{5-6}},
    calory={\unit[3]{kJ}},
    source = {Somebody you used know}
]
{Pan básico}
    
    %\graph
    %{% pictures
    %    small=pic/glass,     % small picture
    %    big=pic/ingredients  % big picture
    %}
    
    \introduction{%
        \blindtext
    }
    
    \ingredients)\\
        3 & Eggs\\
        \unit[200]{ml} & Cream\\
        40 g & Sugar\\
        50 g & Butter
    }
    
    \preparation{%
        \step \blindtext
        \step \blindtext
        \step \blindtext
    }
    
    \suggestion[Headline]
    {%
        \blindtext
    }
    
    \suggestion{%
        \blindtext
    }
    
    \hint{%
        Enjoy typesetting recipes with {\textbf{\Large\LaTeX}} and {\textbf{\Large xcookybooky!}}
    }
    
\end{recipe}
% Complete recipe example
\begin{recipe}
[% 
    preparationtime = {\unit[15]{min}},
    bakingtime={\unit[45]{min}},
    bakingtemperature={\protect\bakingtemperature{
        fanoven=\unit[245]{\textcelcius},
        topbottomheat=\unit[195]{C},
        topheat=\unit[195]{C},
        gasstove=Level 2}},
    portion = {\portion{5-6}},
    calory={\unit[3]{kJ}},
    source = {Somebody you used know}
]
{Pan integral básico}
    
    %\graph
    %{% pictures
    %    small=pic/glass,     % small picture
    %    big=pic/ingredients  % big picture
    %}
    
    \introduction{%
	Esta es la receta de pan que suelo hacer, sin detallar mucho los tiempos porque admite mucho margen en los tiempos. Las cantidades dan para dos hogazas hermosas y la mitad de la primera suele caer para la cena.
    }
    
    \ingredients{%
        \unit[700]{g} Harina integral (T150 Roca) \\
        \unit[300]{g} Harina de fuerza (Mercadona)  \\
        \unit[790]{ml} Agua \\
        \unit[16]{g} Sal \\
        \unit[1]{g} Levadura 
    }
    
    \preparation{%
        \step Mezclar el agua y las harinas sin necesidad de amasar apenas
        \step Tras 15-30 minutos, incorporar la sal y la levadura y amasar un poco
	\step Si se puede, darle unos dobleces a la masa, especialmente en la primera hora
        \step Dejar reposar toda la noche. Por la mañana, dividir en dos, dar forma y guardar en la nevera
	\step Encender el horno unas dos horas antes de necesitar el pan, a \unit[250]{C} con un recipiente pesado dentro.
	\step Cuando el horno esté caliente, en otros 30-40 minutos, poner el pan y tapar
	\step Hornear 20-30 minutos tapado y otros 20-30 minutos destapado.
	\step Dejar enfriar media hora
    }
    
    \suggestion[Puntos clave]
    {%
        Esta masa es bastante elástica y agradece mucho los pliegues y la fermentación lenta. La he hecho con \unit[700]{ml} de agua y \unit[25]{g} de levadura fermentando una hora y queda parecida.
	La harina integral se puede tamizar para filtrar las partes más grandes del salvado. Se pueden añadir a la corteza para no perder su valor integral.
	Más harina integral hace que la masa le cueste mucho subir.
	Si no pierde demasiado gas queda con una miga muy tierna.
    }
    
    \suggestion{%
        Este pan tiene muy buen sabor y sirve para todo, desde las tostadas del desayuno, con mermelada o con tomate, aceite, pimienta y salo para acompañar un guiso.
    }
    
    \hint{%
        Alergias: gluten
    }
    
\end{recipe}
% Complete recipe example
\begin{recipe}
[% 
    preparationtime = {\unit[1]{h}},
    bakingtime={\unit[1]{h}},
    bakingtemperature={\protect\bakingtemperature{
        fanoven=\unit[230]{\textcelcius},
        topbottomheat=\unit[195]{C},
        topheat=\unit[195]{C},
        gasstove=Level 2}},
    portion = {\portion{5-6}},
    calory={\unit[3]{kJ}},
    source = {Somebody you used know}
]
{Pizza}
    
    %\graph
    %{% pictures
    %    small=pic/glass,     % small picture
    %    big=pic/ingredients  % big picture
    %}
    
    \introduction{%
        \blindtext
    }
    
    \ingredients)\\
        3 & Eggs\\
        \unit[200]{ml} & Cream\\
        40 g & Sugar\\
        50 g & Butter
    }
    
    \preparation{%
        \step \blindtext
        \step \blindtext
        \step \blindtext
    }
    
    \suggestion[Headline]
    {%
        \blindtext
    }
    
    \suggestion{%
        \blindtext
    }
    
    \hint{%
        Enjoy typesetting recipes with {\textbf{\Large\LaTeX}} and {\textbf{\Large xcookybooky!}}
    }
    
\end{recipe}

\section{Bibliografía}
Aquí quiero poner referencias bibliográficas pero también poner comentarios sobre cada libro...
\end{document} 
